
\section{\poly\ Approach
\label{approach}}

As we wanted to improve \poly\ page-views and increase unique 
visits to the production portal, we actively avoided during the 
design of the engine to provide recommendations based on 
widely used heuristics such as articles written by same (or similar) 
authors, content in the same category and/or just popular items 
within the portal.
These techniques although effective at times, fail to consider 
a reader's own interests at the time of browsing. In addition, 
prior stated-preferences may be discarded and possible correlations 
among categories might be overlooked.


In \poly, we combine a number of techniques so that we can 
present visitors with suggestions that not only fall within 
the realm of their interests but they also are ``fresh''; 
the notion of being ``fresh'' pertains to pages 
that either have not been read so far or 
may have appeared on the portal data infrastructure recently. 
In this context, a key factor that we had to take into consideration 
was the very large
amount of data collected within a matter of minutes 
in user activity.
Effectively managing the inflow of information as well as 
the outflow of the portal data constitutes a challenge.
In an exploratory phase, we instrumented \vurl\ and measured 
the average dwelling time for a visitor: we found it consistently 
to be around $1$~\emph{second} using {\sl JavaScript} events.
It is within this window of time that recommendations have 
to be compiled and be timely shown to the reader so as 
to increase page-view rates and consequently time spent 
on \vurl.
This timing benchmark also indicated that regardless of the 
sophistication of the recommendation algorithm, if we take 
longer times to generate suggestions this will render 
the engine ineffective.
In this case, the majority of visitors will fail 
to receive both personalized and accurate recommendations 
within the designated real-time slack.
%%
% 
To accommodate the above requirement, \poly's design 
follows a \emph{hybrid} approach 
exploiting a variety of key operational aspects by:
\begin{enumerate}
	\item predominantly focusing on the responsiveness of such an engine,
	\item being able to process important data flows and events real-time,
	\item profiling both users and incoming pieces of content in a timely manner,
	\item deploying effective features from traditional techniques
		used in the \vurl\ engine to this date, 
	\item training on--the--fly as much as possible while 
		relegating off--line work for resource-intensive tasks only,
	\item designing a system that can work in a plug-and-play fashion, 
		regardless of the underlying infrastructure (i.e., database and content 
		management systems used).
\end{enumerate}
All the above features influenced the design of \poly\ and led us to 
deploy an engine that embeds multi-criteria in its core operation 
with the time slack for all jobs taken continuously into account.
In addition, we deploy a fail-over mechanism that addresses 
issues introduced due to new content, users, content category re-alignment, 
and classification re-adjustment.

A number of factors influence the way \poly\ yields its suggestions and include:
\begin{itemize}
\item 
the visitor's unique profile based on user-id and IP number,
\item 	
	articles whose categories are strongly related 
	to content currently in browsing, 
\item 
	the time of day the visitor is browsing the portal,
\item 
	articles that share a high textual--similarity 
	to the one currently being accessed,
\item 
	articles that are popular on this content--category today.
\end{itemize}
By default, each of the above criteria may equally contribute
--in terms of weight--  
to the outcomes computed by \poly. 
In this regard, we seek to ensure the objectiveness of the evaluation 
while we offer the capability 
to appropriately gauge the weighting scheme so as 
to provide content of timely interest. 
Moreover, we want to offer warranties that \poly\ 
regardless of the nature of the visitor and/or the item currently in browsing, 
we can 
locate suggestions even if one or more of the contributing 
criteria fail to produce suggestions.
In this case, the rest of the criteria will kick in 
and help fill in the required quota for compiling 
the list of the suggestions.

We aspire to carry the necessary functionalities in real-time and 
compute all required aspects on-the-fly in a way that the user 
has the time to  view recommendations, evaluate their worthiness 
and likely proceed to read one of those suggestions.
If so, a new batch of recommendations is computed while 
user clicks generate valuable input in term of the navigation 
provided for our engine. 

